\documentclass[twoside]{article}

\usepackage[top=2.5cm,bottom=3cm,left=3cm,right=3cm]{geometry}
\usepackage{graphicx}
%\usepackage{fouriernc}
\usepackage{amssymb}
\usepackage{cite}
\usepackage{float}
\usepackage{environ}
\usepackage[export]{adjustbox} % Left align graphics
\usepackage{tikz}
\usepackage{circuitikz}
\usepackage{siunitx}
\usepackage{montserrat}
\usepackage[sfdefault]{plex-sans}
\usepackage{plex-mono}
\usepackage{caption,footmisc}
\usepackage{fancyhdr}
\usepackage{xcolor}
\usepackage{titlesec}
\usepackage{titling}
\usepackage{longtable}
\usepackage{boldline}
\usepackage{bigstrut}
\usepackage{enumitem}
\usepackage{wrapfig,booktabs}
\usepackage{multirow,array,slashbox}
\usepackage{url}
\usepackage{gensymb}
\usetikzlibrary{shapes, shapes.geometric,arrows,arrows.meta,shapes.arrows,fit,backgrounds}
\frenchspacing

% --------------
% Commands
% --------------
%\renewcommand{\familydefault}{phv}

\newcommand\customrule[1]{
  \color{accent}\vspace{-14pt}{\makebox[\linewidth][l]{\rule{#1\textwidth}{1.5pt}}}
\vspace{-23pt}}

%\renewcommand\labelitemi{--}
\renewcommand{\labelitemi}{\color{accent}\scriptsize$\blacksquare$}

\setlength{\parindent}{0pt}

\newcommand\tss[1]{
  \textsubscript{#1}
}
% --------------
% Definitions
% --------------
%\setlist[itemize]{leftmargin=*,itemsep=0pt}
\newcolumntype{?}{!{\vrule width 1pt}}
\captionsetup{labelfont=bf,size=small}
\newcommand{\component}[1]{\emph{#1}}
\newcommand{\reffigure}[1]{\textit{Figure \ref{#1}}}
\newcommand{\reftable}[1]{\textit{Table \ref{#1}}}
\newcommand{\refapp}[1]{\textit{Appendix \ref{#1}}}
\newcommand{\refsection}[1]{\textit{Section \ref{#1}}}
\newlength{\Oldarrayrulewidth}
\newcommand{\Cline}[2]{%
  \noalign{\global\setlength{\Oldarrayrulewidth}{\arrayrulewidth}}%
  \noalign{\global\setlength{\arrayrulewidth}{#1}}\cline{#2}%
  \noalign{\global\setlength{\arrayrulewidth}{\Oldarrayrulewidth}}}

\definecolor{accent}{HTML}{1967B7}
\definecolor{desat}{HTML}{C2C2C2}

\newcolumntype{L}[1]{>{\raggedright\let\newline\\\arraybackslash\hspace{0pt}}m{#1}}
\newcolumntype{C}[1]{>{\centering\let\newline\\\arraybackslash\hspace{0pt}}m{#1}}
\newcolumntype{R}[1]{>{\raggedleft\let\newline\\\arraybackslash\hspace{0pt}}m{#1}}
\makeatletter 
\def\WFfill{\par 
    \ifx\parshape\WF@fudgeparshape 
    \nobreak 
    \ifnum\c@WF@wrappedlines>\@ne 
    \advance\c@WF@wrappedlines\m@ne 
    \vskip\c@WF@wrappedlines\baselineskip 
    \global\c@WF@wrappedlines\z@ 
    \fi 
    \allowbreak 
    \WF@finale 
    \fi 
} 
\makeatother 
% --------------
% Fancy stuff
% --------------
\titleformat{\section}
            {\Huge\fontfamily{Montserrat-LF}\fontseries{eb}\selectfont}
            {\color{desat}\thesection}
            {0.5em}
            {}
            [\customrule{0.1}]

\titleformat{\subsection}
            {\Large\fontfamily{Montserrat-LF}\fontseries{b}\selectfont}
            {\color{desat}\thesubsection}
            {0.5em}
            {}
            %[\customrule{0.07}]

\titleformat{\subsubsection}
            {\large\fontfamily{Montserrat-LF}\fontseries{b}\selectfont}
            {\thesubsubsection}
            {0.5em}
            {}

\fancyhead[L,C]{}
    \fancyhf{}
    %\fancyfoot[R]{\footnotesize\bfseries\color{accent}Documentation RISC}
    %\fancyhead[LE]{\color{accent}\bfseries\thepage\hskip\linepagesep\vfootline}
    \fancyfoot[R]{\color{desat}\bfseries RISC\hspace{5pt}\vfootline\hskip\linepagesep\color{accent}\bfseries\thepage}
    \renewcommand\headrulewidth{0pt}
    \newskip\linepagesep \linepagesep 5pt\relax
    \def\vfootline{%
    \begingroup\color{desat}\rule[-2pt]{1.5pt}{11pt}\endgroup}

\pagestyle{fancy}



\newcounter{magicrownumbers}
\newcommand\rownumber{\stepcounter{magicrownumbers}\arabic{magicrownumbers}}

% ---------------
% Boxed numbers
% ---------------
\def\scando{}
\def\scan#1{\scanA#1\end}
\def\scanA#1{\ifx\end#1\else\scando#1\expandafter\scanA\fi}

\setlength{\fboxrule}{1pt}
\newcommand{\boxing}[1]{\fbox{\parbox[bottom][0.2cm][l]{0.12cm}{\texttt{\footnotesize#1
\normalsize}}}}
\let\scando\boxing
\NewEnviron{boxednumbers}{\expandafter\expandafter\scan\BODY}

\newcommand{\boxchar}[1]{\begin{boxednumbers} #1
\end{boxednumbers}}

% ---------------
% Document
% ---------------
\begin{document}
\section*{RISC}
\subsection*{Introduction}
The final purpose of this document is not necessarily to be a technical
documentation of the RISC built, but should also be instructional for
other students interested in computer architecture. The goal is to design an
efficient CPU with some intersting and modern features to learn their workings.
\subsection*{Instruction Set}
\subsubsection*{Instructions}
The instruction set consists of 32 different instructions. The first 5 bit of
the operation code are interpreted as the instruction. The following
instructions are implemented:
\begin{table}[H]
	\centering
	\begin{longtable}{p{0.16\textwidth}p{0.05\textwidth}p{0.04\textwidth}p{0.79\textwidth}}
        \textbf{Data transfer} & \rownumber & \texttt{LDW} & Load a value of a memory address into a register\\
                               & \rownumber & \texttt{LDB} & Load the LSB of a memory address into a register\\
                               & \rownumber & \texttt{MOV} & Move a value from one register to another\\
                               & \rownumber & \texttt{MOV} & Move an immediate value in a register\\
                               & \rownumber & \texttt{STW} & Store the value of a register in a memory adress\\
                               & \rownumber & \texttt{STB} & Store the LSB of a register value in a memory address\\
	\end{longtable}
	\begin{longtable}{p{0.16\textwidth}p{0.05\textwidth}p{0.04\textwidth}p{0.79\textwidth}}
        \textbf{Stack} & \rownumber & \texttt{PSH} & Push register values onto the stack\\
                       & \rownumber & \texttt{POP} & Pop values from the stack\
	\end{longtable}
	\begin{longtable}{p{0.16\textwidth}p{0.05\textwidth}p{0.04\textwidth}p{0.79\textwidth}}
        \textbf{ALU} & \rownumber & \texttt{ADD} & Add the values of two registers\\
                     & \rownumber & \texttt{ADD} & Add an immediate value to a register\\
                     & \rownumber & \texttt{SUB} & Subtract the values of two registers\\
                     & \rownumber & \texttt{SUB} & Subtract an immediate value from a register\\
                     & \rownumber & \texttt{MUL} & Multiply the values of two registers\\
                     & \rownumber & \texttt{AND} & Bitwise AND of the values of two registers\\
                     & \rownumber & \texttt{ORR} & Bitwise OR of the values of two registers\\
                     & \rownumber & \texttt{XOR} & Bitwise XOR of the values of two registers\\
                     & \rownumber & \texttt{NOT} & Bitwise inversion of the value in one register\\
                     & \rownumber & \texttt{LLS} & Shift the value in a register left by a variable amount\\
                     & \rownumber & \texttt{ALS} & Shift the value in a register arithmetically left by a variable amount\\
                     & \rownumber & \texttt{RLS} & Rotate the value in a register left by a variable amount\\
                     & \rownumber & \texttt{LRS} & Shift the value in a register right by a variable amount\\
                     & \rownumber & \texttt{ARS} & Shift the value in a register arithmetically right by a variable amount\\
                     & \rownumber & \texttt{RRS} & Rotate the value in a register right by a variable amount\\
                     & \rownumber & \texttt{SXT} & Sign extension of the LSB\\
	\end{longtable}
	\begin{longtable}{p{0.16\textwidth}p{0.05\textwidth}p{0.04\textwidth}p{0.79\textwidth}}
        \textbf{Branching} & \rownumber & \texttt{BRX} & Branch to the address stored in a register\\
                           & \rownumber & \texttt{BIF} & Branch if a certain flag is set in the flag register\\
    \end{longtable}
\end{table}\noindent
The operation codes have the following general layout:\par

\subsubsection*{Architecture}
\renewcommand{\arraystretch}{1.2}
\begin{table}[H]
    \centering
    \begin{longtable}{m{0.30\textwidth}m{0.12\textwidth}m{0.55\textwidth}}
        \textbf{Operation Code} & \textbf{Mnemonic} & \textbf{Description}\\
        \hlineB{2.2}
                                                    \bigstrut
        \texttt{0100 0001 0mmm 1nnn} & LDW       & Load the word that is stored at the
                                                    address held by the register
                                                    \texttt{Rmmm} into the register
                                                    \texttt{Rnnn}.\\
                                                    \hline
                                                    \bigstrut
        \texttt{0100 1001 0mmm 1nnn} & LDB       & Load the least significant byte of
                                                    the word at the address held by the
                                                    register \texttt{Rmmm} into the
                                                    register \texttt{Rnnn}.\\
                                                    \hline
                                                    \bigstrut
        \texttt{0101 0nnn bbbb bbbb} & MOVE      & Move the value
                                                    \texttt{bbbb'bbbb} into the register
                                                    \texttt{Rnnn}.\\
                                                    \hline
                                                    \bigstrut
        \texttt{0110 0001 0mmm 1nnn} & STRW      & Store the value of the register
                                                    \texttt{Rmmm} at the address held by
                                                    the register \texttt{Rnnn}.\\
                                                    \hline
                                                    \bigstrut
        \texttt{0110 1001 0mmm 1nnn} & STRB      & Store the least significant byte of
                                                    the value in register
                                                    \texttt{Rmmm} at the address held by
                                                    the register \texttt{Rnnn}.\\
                                                    \hline
                                                    \bigstrut
        \texttt{0000 00ps xxxx xxxx} & PUSH      & Push the declared registers \texttt{x}, the stack pointer \texttt{s} or the program counter \texttt{p} on the stack\\
        \hline
        \texttt{0000 10ps xxxx xxxx} & POP       & Pop values from the stack in the declared registers \texttt{x}, the stack pointer \texttt{s} or the program counter \texttt{p}.\\
        \hline
        \texttt{1000 001m mmnn nddd} & ADD       & Add the values of the registers
                                                    \texttt{Rmmm} and \texttt{Rnnn} and
                                                    store the result in the register
                                                    \texttt{Rddd}.\\
                                                    \hline
                                                    \bigstrut
        \texttt{1000 101m mmnn nddd} & SUB       & Subtract the value in the register
                                                    \texttt{Rnnn} from the value in
                                                    the register \texttt{Rmmm} and store
                                                    the result in the register
                                                    \texttt{Rddd}.\\
                                                    \hline
                                                    \bigstrut
        \texttt{1001 001m mmnn nddd} & AND       & Calculate a bitwise AND of the values in
                                                    the registers \texttt{Rmmm}
                                                    and \texttt{Rnnn}. Store the result
                                                    in the register \texttt{Rddd}.\\
                                                    \hline
                                                    \bigstrut
        \texttt{1001 101m mmnn nddd} & OR        & Calculate a bitwise OR of the values
                                                    in the registers \texttt{Rmmm} and
                                                    \texttt{Rnnn}. Store the result in
                                                    the register \texttt{Rddd}.\\
                                                    \hline
                                                    \bigstrut
        \texttt{1010 0001 0mmm 1ddd} & NOT       & Bitwise inversion of the value in
                                                    the register \texttt{Rmmm}. Store
                                                    the result in the register
                                                    \texttt{Rddd}.\\
                                                    \hline
                                                    \bigstrut
        \texttt{1010 1001 0mmm bbbb} & LSHIFT    & Shift the value in the register
                                                    \texttt{Rmmm} by \texttt{bbbb} to
                                                    the left.\\
                                                    \hline
                                                    \bigstrut
        \texttt{1011 0001 0mmm bbbb} & RSHIFT    & Shift the value in the register
                                                    \texttt{Rmmm} by \texttt{bbbb} to
                                                    the right.\\
                                                    \hline
                                                    \bigstrut
        \texttt{1011 1000 0000 1nnn} & SIXT      & Signextend the least significant
                                                    byte of the value that is stored
                                                    in the register \texttt{Rnnn}.\\
                                                    \hline
    \end{longtable}
\end{table}

\subsection*{ALU}
The ALU consists of the following components:
\begin{itemize}
    \item Brent-Kung Adder (BKA)
    \item Barrel Shifter
    \item Dadda Tree Multiplier (DTM)
    \item Logic Unit (LU)
\end{itemize}

\subsubsection*{Brent-Kung Adder}
A Brent-Kung adder is a parallel prefix adder.\cite{brent82}

\subsubsection*{Barrel Shifter}
The barrel shifter is able to (arithmetically) shift and rotate all the bits
to the left or to the right. The design was taken from Pillmeier et
al.\cite{pillmeier02}.

\subsubsection*{Dadda Tree Mulitplier}
A Dadda Tree Multiplier uses a Dadda Tree to multiply two numbers.

\bibliography{refs}
\bibliographystyle{plain}
\end{document}
